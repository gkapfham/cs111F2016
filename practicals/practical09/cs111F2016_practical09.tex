% CS 111 style
% Typical usage (all UPPERCASE items are optional):
%       \input 111pre
%       \begin{document}
%       \MYTITLE{Title of document, e.g., Lab 1\\Due ...}
%       \MYHEADERS{short title}{other running head, e.g., due date}
%       \PURPOSE{Description of purpose}
%       \SUMMARY{Very short overview of assignment}
%       \DETAILS{Detailed description}
%         \SUBHEAD{if needed} ...
%         \SUBHEAD{if needed} ...
%          ...
%       \HANDIN{What to hand in and how}
%       \begin{checklist}
%       \item ...
%       \end{checklist}
% There is no need to include a "\documentstyle."
% However, there should be an "\end{document}."
%
%===========================================================
\documentclass[11pt,twoside,titlepage]{article}
%%NEED TO ADD epsf!!
\usepackage{threeparttop}
\usepackage{graphicx}
\usepackage{latexsym}
\usepackage{color}
\usepackage{listings}
\usepackage{fancyvrb}
%\usepackage{pgf,pgfarrows,pgfnodes,pgfautomata,pgfheaps,pgfshade}
\usepackage{tikz}
\usepackage[normalem]{ulem}
\tikzset{
    %Define standard arrow tip
%    >=stealth',
    %Define style for boxes
    oval/.style={
           rectangle,
           rounded corners,
           draw=black, very thick,
           text width=6.5em,
           minimum height=2em,
           text centered},
    % Define arrow style
    arr/.style={
           ->,
           thick,
           shorten <=2pt,
           shorten >=2pt,}
}
\usepackage[noend]{algorithmic}
\usepackage[noend]{algorithm}
\newcommand{\bfor}{{\bf for\ }}
\newcommand{\bthen}{{\bf then\ }}
\newcommand{\bwhile}{{\bf while\ }}
\newcommand{\btrue}{{\bf true\ }}
\newcommand{\bfalse}{{\bf false\ }}
\newcommand{\bto}{{\bf to\ }}
\newcommand{\bdo}{{\bf do\ }}
\newcommand{\bif}{{\bf if\ }}
\newcommand{\belse}{{\bf else\ }}
\newcommand{\band}{{\bf and\ }}
\newcommand{\breturn}{{\bf return\ }}
\newcommand{\mod}{{\rm mod}}
\renewcommand{\algorithmiccomment}[1]{$\rhd$ #1}
\newenvironment{checklist}{\par\noindent\hspace{-.25in}{\bf Checklist:}\renewcommand{\labelitemi}{$\Box$}%
\begin{itemize}}{\end{itemize}}
\pagestyle{threepartheadings}
\usepackage{url}
\usepackage{wrapfig}
\usepackage{hyperref}
%=========================
% One-inch margins everywhere
%=========================
\setlength{\topmargin}{0in}
\setlength{\textheight}{8.5in}
\setlength{\oddsidemargin}{0in}
\setlength{\evensidemargin}{0in}
\setlength{\textwidth}{6.5in}
%===============================
%===============================
% Macro for document title:
%===============================
\newcommand{\MYTITLE}[1]%
   {\begin{center}
     \begin{center}
     \bf
     CMPSC 111\\Introduction to Computer Science I\\
     Spring 2016\\
     \medskip
     \end{center}
     \bf
     #1
     \end{center}
}
%================================
% Macro for headings:
%================================
\newcommand{\MYHEADERS}[2]%
   {\lhead{#1}
    \rhead{#2}
    \immediate\write16{}
    \immediate\write16{DATE OF HANDOUT?}
    \read16 to \dateofhandout
    \lfoot{\sc Handed out on \dateofhandout}
    \immediate\write16{}
    \immediate\write16{HANDOUT NUMBER?}
    \read16 to\handoutnum
    \rfoot{Handout \handoutnum}
   }

%================================
% Macro for bold italic:
%================================
\newcommand{\bit}[1]{{\textit{\textbf{#1}}}}

%=========================
% Non-zero paragraph skips.
%=========================
\setlength{\parskip}{1ex}

%=========================
% Create various environments:
%=========================
\newcommand{\PURPOSE}{\par\noindent\hspace{-.25in}{\bf Purpose:\ }}
\newcommand{\SUMMARY}{\par\noindent\hspace{-.25in}{\bf Summary:\ }}
\newcommand{\DETAILS}{\par\noindent\hspace{-.25in}{\bf Details:\ }}
\newcommand{\HANDIN}{\par\noindent\hspace{-.25in}{\bf Hand in:\ }}
\newcommand{\SUBHEAD}[1]{\bigskip\par\noindent\hspace{-.1in}{\sc #1}\\}
%\newenvironment{CHECKLIST}{\begin{itemize}}{\end{itemize}}

\begin{document}

\MYTITLE{Practical 9 \\ November 18, 2016 \\ Due in Bitbucket by midnight on the day of your practical \\ ``Checkmark'' grade}

\vspace*{-.2in}
\subsection*{Summary}
\vspace*{-.05in}

In this practical assignment, you will explore an already implemented Java program that causes various types of
exceptions to be thrown. As you modify the program, you will uncomment certain sections of it and then observe, record,
and comment on the output that is produced. In addition, you will add some source code to the program that causes a new
type of exception to be thrown. Finally, you will continue to practice using a Git repository hosted by Bitbucket.

\vspace*{-.15in}
\subsection*{Review the Textbook}
\vspace*{-.05in}

To learn more about the concepts associated with exception handling in the Java programming language, please study the
content in Chapter 11. In particular, make sure that you understand the purpose of exceptions, the messages displayed in
the terminal when an exception is uncaught, and the structure and meaning of the {\tt try-catch-finally} blocks used in
Java programs.

\vspace*{-.15in}
\subsection*{Using and Extending a Program that Throws Exceptions}
\vspace*{-.05in}

Please return to the ``share'' repository for this course and type the command ``{\tt git pull}''. Now, please find the
``{\tt practical09/}'' directory and study this source code. After reviewing the provided source code file, please
compile and run the ``{\tt ExceptionExample.java}'' program. At the outset, you will see that this program does not
throw any exceptions at all. So, what you should do next is to incrementally uncomment each of the calls to the {\tt
throwsExceptions} method and observe and record the output from the program. Each time the program throws a different
exception, make sure that you understand why it does so. Next, you will notice that, in certain cases, the program does
not output a full ``stack trace'' that prints out in the terminal window which exception is thrown. As such, you should
add code to the catch blocks of {\tt ExceptionExample} that can print the stack trace. Next, you must add a new code
segment that will cause the program to throw a new type of exception. Please submit the modified, and fully commented,
version of the {\tt ExceptionExample} in your own Git repository. Finally, you must create a file called {\tt
exception\_output} that records the output from your various runs of the program; please include a run showing your new
exception.

\vspace*{-.15in}
\subsection*{General Guidelines for Practical Sessions}
\vspace*{-.05in}

\begin{itemize}
  \itemsep 0in

\item {\bf Submit \textbf{\textit{Something}}.} Your grade for this assignment is a ``checkmark'' indicating whether you
  did or did not complete the work and submit something to the Bitbucket repository.

\item {\bf Update Your Repository Often!} You should {\tt add}, {\tt commit}, and {\tt push} your updated files each
  time you work on them, always including descriptive messages about each code change.

\item {\bf Review the Honor Code Policy on the Syllabus.} Please remember that source code that is nearly identical to
  the work of others will be taken as evidence of violating the \mbox{Honor Code}.

\end{itemize}

\end{document}
