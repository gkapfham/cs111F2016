% CS 111 style
% Typical usage (all UPPERCASE items are optional):
%       \input 111pre
%       \begin{document}
%       \MYTITLE{Title of document, e.g., Lab 1\\Due ...}
%       \MYHEADERS{short title}{other running head, e.g., due date}
%       \PURPOSE{Description of purpose}
%       \SUMMARY{Very short overview of assignment}
%       \DETAILS{Detailed description}
%         \SUBHEAD{if needed} ...
%         \SUBHEAD{if needed} ...
%          ...
%       \HANDIN{What to hand in and how}
%       \begin{checklist}
%       \item ...
%       \end{checklist}
% There is no need to include a "\documentstyle."
% However, there should be an "\end{document}."
%
%===========================================================
\documentclass[11pt,twoside,titlepage]{article}
%%NEED TO ADD epsf!!
\usepackage{threeparttop}
\usepackage{graphicx}
\usepackage{latexsym}
\usepackage{color}
\usepackage{listings}
\usepackage{fancyvrb}
%\usepackage{pgf,pgfarrows,pgfnodes,pgfautomata,pgfheaps,pgfshade}
\usepackage{tikz}
\usepackage[normalem]{ulem}
\tikzset{
    %Define standard arrow tip
%    >=stealth',
    %Define style for boxes
    oval/.style={
           rectangle,
           rounded corners,
           draw=black, very thick,
           text width=6.5em,
           minimum height=2em,
           text centered},
    % Define arrow style
    arr/.style={
           ->,
           thick,
           shorten <=2pt,
           shorten >=2pt,}
}
\usepackage[noend]{algorithmic}
\usepackage[noend]{algorithm}
\newcommand{\bfor}{{\bf for\ }}
\newcommand{\bthen}{{\bf then\ }}
\newcommand{\bwhile}{{\bf while\ }}
\newcommand{\btrue}{{\bf true\ }}
\newcommand{\bfalse}{{\bf false\ }}
\newcommand{\bto}{{\bf to\ }}
\newcommand{\bdo}{{\bf do\ }}
\newcommand{\bif}{{\bf if\ }}
\newcommand{\belse}{{\bf else\ }}
\newcommand{\band}{{\bf and\ }}
\newcommand{\breturn}{{\bf return\ }}
\newcommand{\mod}{{\rm mod}}
\renewcommand{\algorithmiccomment}[1]{$\rhd$ #1}
\newenvironment{checklist}{\par\noindent\hspace{-.25in}{\bf Checklist:}\renewcommand{\labelitemi}{$\Box$}%
\begin{itemize}}{\end{itemize}}
\pagestyle{threepartheadings}
\usepackage{url}
\usepackage{wrapfig}
\usepackage{hyperref}
%=========================
% One-inch margins everywhere
%=========================
\setlength{\topmargin}{0in}
\setlength{\textheight}{8.5in}
\setlength{\oddsidemargin}{0in}
\setlength{\evensidemargin}{0in}
\setlength{\textwidth}{6.5in}
%===============================
%===============================
% Macro for document title:
%===============================
\newcommand{\MYTITLE}[1]%
   {\begin{center}
     \begin{center}
     \bf
     CMPSC 111\\Introduction to Computer Science I\\
     Spring 2016\\
     \medskip
     \end{center}
     \bf
     #1
     \end{center}
}
%================================
% Macro for headings:
%================================
\newcommand{\MYHEADERS}[2]%
   {\lhead{#1}
    \rhead{#2}
    \immediate\write16{}
    \immediate\write16{DATE OF HANDOUT?}
    \read16 to \dateofhandout
    \lfoot{\sc Handed out on \dateofhandout}
    \immediate\write16{}
    \immediate\write16{HANDOUT NUMBER?}
    \read16 to\handoutnum
    \rfoot{Handout \handoutnum}
   }

%================================
% Macro for bold italic:
%================================
\newcommand{\bit}[1]{{\textit{\textbf{#1}}}}

%=========================
% Non-zero paragraph skips.
%=========================
\setlength{\parskip}{1ex}

%=========================
% Create various environments:
%=========================
\newcommand{\PURPOSE}{\par\noindent\hspace{-.25in}{\bf Purpose:\ }}
\newcommand{\SUMMARY}{\par\noindent\hspace{-.25in}{\bf Summary:\ }}
\newcommand{\DETAILS}{\par\noindent\hspace{-.25in}{\bf Details:\ }}
\newcommand{\HANDIN}{\par\noindent\hspace{-.25in}{\bf Hand in:\ }}
\newcommand{\SUBHEAD}[1]{\bigskip\par\noindent\hspace{-.1in}{\sc #1}\\}
%\newenvironment{CHECKLIST}{\begin{itemize}}{\end{itemize}}

\begin{document}
\MYTITLE{Practical 3\\February 5, 2016\\Due in Bitbucket by midnight of the day of your practical \\ ``Checkmark'' grade}

\vspace*{-.3in}
\subsection*{Summary}

In this assignment, you will ``seed'' one of your past programs with defects, share this defective program with a
partner (who will also share a program with you), and then you both will see if you can find the mistakes that were
purposefully introduced. This assignment will allow you to further practice the skills of debugging programs and orally
expressing your intentions for a program.

\vspace*{-.15in}
\subsection*{Review the Textbook}
\vspace*{-.075in}

In addition to studying the slides for Chapters 1 and 2, be sure to read Sections 2.1 through 2.6 of your book as it
will give you a good review of the content that we have studied so far. Also, you should briefly review all of the Java
programs that you have already written for this class --- when you are looking for this program, make sure that you look
into both the ``share'' repository for this course and the ``student'' repository that you used to submit your
assignments. Please see the course instructor if you have questions about topics like variables, expressions, and
operators.

\vspace*{-.15in}
\subsection*{Seed Defects into Your Program}
\vspace*{-.075in}

Before you start creating the Java program required by this assignment, you should separately type the commands ``{\tt
cd cs111S2016}'', ``{\tt cd cs111S2016-<your user name>}'', and ``{\tt cd practicals}'' in your terminal window. Once
you are in the {\tt practicals/} directory of your Git repository, you can type the command ``{\tt mkdir practical03}''
to create a new directory for this assignment. Later, you can run the {\tt gvim} command from this directory when you
are ready to begin to modify the required Java program. Please see the instructor if you have problems with these
preparatory steps.

Once you have found a program that you have already implemented for this class, review its source code to ensure that
you can remember your motivation for programming it the way that you did. Next, copy this program into the {\tt
practical03/} directory and then take time to fully comment the code so that it explains all of your intentions. At this
stage of the assignment, you can practice using the two commenting standards that are available in the Java programming
language. Please use the {\tt javac} and {\tt git} command regularly so that, as soon as you add new comments to the
program, you try to re-compile it and, if that works correctly, you commit the new version to your repository. Please
see the instructor if you are not sure how to work incrementally in this fashion.

Now you are invited to ``seed'' five different types of defects into your chosen Java program. As you are adding these
defects, please make sure that your partner cannot see the mistakes that you are making. You should aim to insert two
types of ``bugs'' into your program, with the goal of having five defects in total. At least three or four of your defects
should be ones that will be detected by the Java compiler. The other one or two defects should be ``logic mistakes''
that will be accepted by the Java compiler and yet lead to a program that does not produce the type of output that you
describe in your program's comments. Please try to add defects that are similar to the mistakes that you have been
making during our class and laboratory assignments. While the mistakes should not be overly obvious, they should also
reflect the types of errors commonly made by programmers in Computer Science 111. You should take notes as to where you
placed the mistakes, taking care to ensure that you partner does not see these notes.

Once you and your partner are done inserting defects into your programs, you should explain to each other the
overarching purpose of the programs that you implemented. For instance, if your program produces a diagram in the
terminal window, then you should explain what it should look like and why you decided to implement it the fashion that
you did. Or, if your program serves an intended purpose (e.g., a menu system for a tip calculator), then you should
detail it and highlight your goals for implementing the entire system. As you are discussing these matters with your
partner, please be as detailed and specific as is possible, expressing all of the relevant aspects of your creation. Of
course, you should be careful to not reveal the location of your program's defects!

\vspace*{-.15in}
\subsection*{Finding and Fixing Seeded Defects}
\vspace*{-.075in}

After you have inserted the defects into your program and you have had a thorough conversation with your partner, you
should use email or Slack to share the source code of your program with each other. Please note that students should
only use email or Slack for code sharing and not, under any circumstances, give their partner access to their entire
Bitbucket repository. Once you have received a copy of your partner's file, please make sure that you save it in the
{\tt practical03/} directory of your Git repository. Now, try to find and fix all of the defects that your partner
inserted into the program. Once you locate and resolve one of these issues, please put a comment into the code to
explain the problem that you found and the way in which you decided to handle it. As you fix the ``bugs'', you should
use {\tt git} to transfer the improved version of the program to Bitbucket. Ultimately, your {\tt practical03/}
directory in your Git repository should contain the source code for your defective program and the corrected source code
for your partner's program. Finally, you should use {\tt gvim} to create a one to three paragraph file, called {\tt
response}, that articulates and reflects on your experiences in seeding and finding defects in Java programs.

\vspace*{-.15in}
\subsection*{General Guidelines for Practical Sessions}
\vspace*{-.075in}

As you are typing your program in the {\tt gvim} text editor, you should regularly save your files.  Once you have
created a preliminary version of your program and it compiles and runs as anticipated, you should use the ``{\tt git
add}'' command to ``stage'' it in your Git repository.  Next, you can use the ``{\tt git commit}'' command to save it in
your local repository with a version control message.  Finally, you can run ``{\tt git push}'' to transfer your file to
the Bitbucket servers.  For this practical assignment, you do {\em not} have to hand in a hard copy of anything---just
upload your Java programs and the ``response'' document to Bitbucket by using the appropriate {\tt git} commands.
Please review your ``Git Cheatsheet'' and talk with a member of the class, the course instructor, or a teaching
assistant if you do not understand how to use some aspect of the Git version control system. Don't forget that it is
extremely important for you to keep your Bitbucket repository well organized. Also, you are responsible for ensuring
that your repository is shared with the course instructor.

Since this is one of our first practical assignments and you are still learning how to use the Java programming
language, don't become frustrated if you make a mistake. Instead, use your mistakes as an opportunity for learning about
the needed knowledge and skills in computer science.

% technology and the background and expertise of the other students in the class, the teaching
% assistants, and the instructor.

% Please follow these general guidelines as you complete this practical assignment:

% \vspace*{-.05in}
% \begin{itemize}

% % \itemsep -0.01in
% \itemsep 0in

% \item {\bf Experiment!} Practical sessions are for learning by doing without the pressure of grades or ``right/wrong''
%   answers. So try things!  The best way to learn is by intelligently experimenting.

% \item {\bf Submit \textbf{\textit{Something}}.} Your grade for this assignment is a ``checkmark'' indicating whether you
%   did or did not complete the work and submit something to the Bitbucket repository using the ``{\tt git add}'', ``{\tt
%     git commit}'', and ``{\tt git push}'' commands.

% \item {\bf Practice Key Laboratory Skills.} As you are completing this assignment, practice using the {\tt gvim} text
%   editor and the Ubuntu terminal until you can easily use their most important features.  Additionally, ask
%   a teaching assistant or the course instructor to teach you some of the advanced features of {\tt gvim} and the
%   terminal, thereby helping you to work more effectively.

% \item {\bf Try to Finish During the Class Session.} Practical exercises are not intended to be the equal of the
%   laboratory assignments. If you are simply a slow typist, I've given you until the end of the day, but ideally you
%   should upload a file, even a partially working one, by the end of the class period. So, make sure that you correctly
%   upload your file to your Git repository!

%   % You also should ensure that, for this assignment and all subsequent assignments, you can confidently upload files to
%   % your Git repository during the practical session.

% \item {\bf Help One Another!} If your neighbor is struggling and you know what to do, offer your help. Don't ``do the
%   work'' for them, but advise them on what to type or how to handle things. If you are stuck on a part of this practical
%   session and you could not find any insights in either your textbook or online sources, formulate an intelligent
%   question to ask your neighbor, a teaching assistant, or a course instructor. Try to strike the right balance between
%   asking for help when you cannot solve a problem and working independently to find a solution.

% \item {\bf Update Your Repository Often!} You should {\tt add}, {\tt commit}, and {\tt push} your updated files each
%   time you work on them, always including descriptive messages about each code change.

% \item {\bf Review the Honor Code Policy on the Syllabus.} Remember that while you may discuss programs with other
%   students in the course, programs that are nearly identical to, or merely variations on, the work of others will be
%   taken as evidence of violating the Honor Code.

% \end{itemize}

\end{document}
